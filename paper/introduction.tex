%!TEX root = main.tex

\section{Introduction}
\label{sec:intro}

During recent years, \gh (2008) has become the largest code host in the world, with more than 5M developers
collaborating across 10M repositories.
Due to its support for distributed version control (Git) and pull-based development~\cite{barr2012cohesive}, 
as well as its modern Web UI and focus on social coding~\cite{dabbish2012social}, \gh has surpassed in size
and popularity even much older forges such as Sourceforge (1999).
As a result, numerous projects (especially open source) are migrating their code base to \gh (for instance, 
the Google query \emph{migrate to github} returns more than 4M results), which now hosts popular projects
such as Ruby on Rails, Homebrew, Bootstrap, Django or JQuery.

Researchers have quickly jumped on board and have started exploring the richness of \gh data.
So far, studies focused on 
building language models of source code~\cite{allamanis2013mining}, 
understanding the effects of branching and pull-based software development~\cite{lee2013git, gousios2014exploratory}, 
uncovering associations between crowdsourced knowledge and software development~\cite{vasilescu2013stackoverflow},
visualizing collaboration and influence~\cite{heller2011visualizing}, 
exploring the social network of developers~\cite{thung2013network, schall2013follow, jiang2013understanding},
or investigating how the social nature of \gh impacts collaboration~\cite{dabbish2012social, marlow2013impression}
and could be used to improve development practices~\cite{pham2013creating, pham2013building}.

To facilitate studies of \gh, we have created \ght~\cite{gousios2012ghtorrent, gousios2013ghtorent}, a scalable, 
queriable, offline mirror of the data offered through the \gh REST API.
\ght data has already been used in empirical studies (e.g., \cite{gousios2014exploratory, squire2014forge, 
vasilescu2013stackoverflow}), and a subset of it has been selected as the topic of the Mining Challenge
at the 2014 edition of the Working Conference on Mining Software Repositories.

In this paper we present the extensions brought to \ght since its official release~\cite{gousios2013ghtorent},
designed to offer customisable data dumps on demand.
The new \ght data-on-demand service offers several advantages.
First, while the \ght project already offered data dumps of both its raw data (MongoDB, currently more than 2TB) 
and up to date metadata (MySQL, currently more than 20GB), downloading and restoring these dumps can be 
very time consuming and might not be necessary if a particular analysis is restricted in scope to say a handful
of ``interesting'' \gh projects (e.g., the Ruby on Rails project, for which separate data sets started being 
collected~\cite{wagstrom2013network}).

%downloadable versions of the MongoDB raw dataset, downloading and restoring them to MongoDB can be very time consuming. For this reason, we have created a publicly available version of the data as they are collected by our main MongoDB server. The data is always updated to the latest available MongoDB dump. The only prerequisite is to have the MongoDB client and SSH installed on your machine.
%
%Obtaining access
%Send us your public SSH key (usually in ~/.ssh/id_rsa.pub)
%When we contact you back, you will be able to setup an SSH tunnel with the following command: ssh -L 27017:dutiht.st.ewi.tudelft.nl:27017 ghtorrent@dutiht.st.ewi.tudelft.nl. Keep in mind that no shell will be allocated in the open SSH session.
%You will then be able to connect to our server using the command: mongo github. There is no password, but please make sure that YOU DO NOT DELETE ANY DATA. Deleting data will only harm other users of the system, which will have to wait for a full restore. The real live database is stored on another machine.
%Collections available in MongoDB
%Have a look here.
%
%Things to keep in mind
%The hosting machine, while powerful, is not capable of processing the data very quickly. At the time of this writing, the data is more than 1.9TB.
%Other people may be using the machine as well. Make sure that you do not run very heavy queries.
%At any time the machine may become unavailable.
%Some data may be missing; if you are willing to provide workers to collect them, please contact us.
%The data is provided in kind to help other people to do research with. Please do not abuse the service.
%The data is offered as is without any explicit or implicit quality or service guarantee from our part.
%All operations are logged for security purposes.

%Github is a dynamic site where developers, projects and wikis are created and deleted constantly.
%\cite{gonzalez2012reproducibility}

%, an effort to create a scalable, queriable, offline mirror of data offered through the Github REST API.



%An outline of the project and bits of the implementation were presented in [2]. Since this work, we extended the collection process to an additional 15 API end points, stabilized the data and metadata schema and developed a service to collaboratively collect and share data. More than 900GB of raw data and 10GB of metadata have been collected and are available for download. In this paper, we present the finalized schema, go through the challenges and limitations of working with the dataset and outline research opportunities that emerge from it.

%works, first on the GHTorrent corpus and then on a carefully selected sample of 291 projects. We find that the pull request model offers fast turnaround, increased opportunities for community engagement and decreased time to incorporate contributions \cite{gousios2014exploratory}

%Visualizing collaboration and influence in the open-source software community \cite{heller2011visualizing}
%transparency and collaboration \cite{dabbish2012social}
%Impression formation in online peer production: activity traces and personal profiles \cite{marlow2013impression}
%language model of source code, based on 352 million lines of Java \cite{allamanis2013mining}


%A number of software engineering research studies have already used the artifacts from Github in interesting ways, see [10][11][12][13]. These studies are part of the research community's history of making tools to study the "usual data sources" [14] of the software development process. For example in [10] the authors combine a social graph of Github users with their commit and follow actions. They then use the geographic data in the user profile to geolocate the users and make inferences about influence in the community.