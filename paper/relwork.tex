%!TEX root = main.tex

\section{Related work}
\label{sec:relwork}

The idea of providing or retrieving software repository data on-demand as such is not new
and can be seen as related to ``Data as a Service'' or ``Information as a Service''~\cite{Dan2007IaaS}.
The data being provided was usually limited to the meta-data~\cite{Codebook} or elements of the
repository such as files~\cite{Voinea2006Mining}.
Similarly to the latter work, lean \ght provides elements of \gh.
However, \gh is a \emph{repository of repositories}~\cite{Sowe2007Using} or
\emph{meta repository}~\cite{Gruhn2013Security} and, therefore, its elements are repositories themselves.
Meta repositories, including lean \ght, provide for cross-domain analysis~\cite{Sowe2007Using}.
As opposed to existing meta repositories such as \ohloh or \flossmole~\cite{Howisom2006FLOSSmole}, % or \ossmetrics~\cite{},
lean \ght provides the researchers with the possibility to select their own object of study rather than
being forced to analyse the entire collection searching for the proverbial needle.% in a haystack.
Moreover, as opposed to such efforts as Boa~\cite{Dyer-Nguyen-Rajan-Nguyen-13} integrating the repository analysis 
tasks in the web-based interface, lean \ght allows researchers to download the relevant repositories
and subject them to further processing by independent tools, i.e., the analysis tasks are not restricted to
the functionality provided by the web-interface.

Projects hosted \gh or the entire \gh collection have been subject to numerous studies (e.g., \cite{allamanis2013mining,dabbish2012social,heller2011visualizing,%
gousios2014exploratory,jiang2013understanding,
lee2013git,marlow2013impression,pham2013creating,
pham2013building,%
%pletea2014security,
schall2013follow,
thung2013network}).
The \gh mirror and the predecessor of the current work~\cite{gousios2012ghtorent},
has also been used in empirical studies~\cite{squire2014forge,vasilescu2013stackoverflow}.

